\documentclass[a4paper,twoside]{memoir}
\usepackage[utf8]{inputenc}
\usepackage[british]{babel}
\usepackage{csquotes}
\usepackage[T1]{fontenc}
\usepackage{charter}
\usepackage[bitstream-charter]{mathdesign}
\usepackage[final,babel]{microtype}
\usepackage[hidelinks,pdfpagelayout=TwoPageRight]{hyperref}
\usepackage{xcolor}
\usepackage{menukeys}
\usepackage{minted}
\usepackage{amsmath}

\newcommand{\FrameTitle}[2]{%
  \fboxrule=\FrameRule \fboxsep=\FrameSep
  \fbox{\vbox{\nobreak \vskip -0.7\FrameSep
    \rlap{\centerline{\strut#1}}\nobreak\nointerlineskip% centered
    \vskip 0.7\FrameSep
    \hbox{#2}}}}
\newenvironment{framewithtitle}[2][\FrameFirst@Lab\ (cont.)]{%
  \def\FrameFirst@Lab{\textbf{#2}}%
  \def\FrameCont@Lab{\textbf{#1}}%
  \def\FrameCommand##1{%
    \FrameTitle{\FrameFirst@Lab}{##1}}%
  \def\FirstFrameCommand##1{%
    \FrameTitle{\FrameFirst@Lab}{##1}}%
  \def\MidFrameCommand##1{%
    \FrameTitle{\FrameCont@Lab}{##1}}%
  \def\LastFrameCommand##1{%
    \FrameTitle{\FrameCont@Lab}{##1}}%
\MakeFramed{\advance\hsize-\width \FrameRestore}}%
{\endMakeFramed}

\newcounter{exercisectr}
\newenvironment{exercise}
{\stepcounter{exercisectr}\begin{framewithtitle}{Practical \arabic{exercisectr}}}
{\end{framewithtitle}}

\newcommand{\shellcmd}{\texttt}
\newcommand{\shellvar}[1]{$\langle \text{#1}\rangle$}
\newcommand{\additional}{\medskip\noindent{\textit{Additional exercises}}}

\begin{document}
Linux offers a command line (also known as a `terminal' or `shell') which allows you to manipulate files, process data, and perform almost any other operation
Many built-in commands accept input from, and print output to, the command line
Use \keys{ctrl+c} to abort a command that is currently running
Use \keys{ctrl+d} on a blank line when you have finished entering all the input to a command
Most commands come with manuals, called `man pages'; if you know the name of the command, \shellcmd{man \shellvar{mycommand}} displays its man page.  Type \keys{q} to return to the command line.

\begin{exercise}
Try sorting a list of numbers.  The \shellcmd{sort} command takes its input from the command line and expects each list item on a new line.  Try entering the following
\begin{minted}{bash}
$ sort --numeric-sort
11.4
11.2
-2
4
\end{minted}
Type \keys{\ctrl+d} on an empty line when you have finished entering the list.  The result should be printed to the command line.

If you make a mistake entering the data, you can abort the \shellcmd{sort} command by typing \keys{\ctrl+c}.  Notice that no sorting is performed and nothing is printed to the command line.

\additional

In the previous exercise we used the \shellcmd{-{}-numeric-sort} option to sort items numerically.  The \shellcmd{sort} command has many other options that control the way data is sorted.

Try sorting words in reverse alphabetical order, ignoring upper and lower cases.  To find the appropriate options, take a look at the manual page with \shellcmd{man sort}, and type \keys{q} to get back to the command line prompt.

The output of one command can be used as the input to another using a pipe (\shellcmd{|}).  To remove duplicate items in a list, we can sort them, then remove adjacent lines that are the same.  Try the following command
\begin{minted}{bash}
$ sort | uniq
bob
alice
charlie
bob
charlie
\end{minted}
using \keys{ctrl+d} when you have finished entering the list.
\end{exercise}
\end{document}
